\begin{titlepage}
  \begin{center}

  {\Huge AXIS\_UART}

  \vspace{25mm}

  \includegraphics[width=0.90\textwidth,height=\textheight,keepaspectratio]{img/AFRL.png}

  \vspace{25mm}

  \today

  \vspace{15mm}

  {\Large Jay Convertino}

  \end{center}
\end{titlepage}

\tableofcontents

\newpage

\section{Usage}

\subsection{Introduction}

\par
Simple UART core for TTL rs232 software mode data communications. No hardware handshake.
This contains its own internal baud rate generator that creates an enable to allow data output
or sampling. Baud clock and aclk can be the same clock.

RTS/CTS is not implemented, it simply asserts it as if its always ready, and ignores CTS.

\subsection{Dependencies}

\par
The following are the dependencies of the cores.

\begin{itemize}
  \item fusesoc 2.X
  \item iverilog (simulation)
  \item cocotb (simulation)
\end{itemize}

\input{src/fusesoc/depend_axi_lite_uart.tex}

\input{src/fusesoc/depend_wishbone_classic_uart.tex}

\input{src/fusesoc/depend_up_uart.tex}

\subsection{In a Project}
\par
First, pick a core that matches the target bus in question. Then connect the BUS UART core to that bus. Once this is complete the UART pins will need
to be routed so they match the UART device or other.

\section{Architecture}
\par
This core is made up of other cores that are documented in detail in there source. The cores this is made up of are the,
\begin{itemize}
  \item \textbf{axis\_uart} Interface with UART and present the data over AXIS interface (see core for documentation).
  \item \textbf{fifo} Used for RX and TX FIFO instances. Set to 16 words buffer max (see core for documentation).
  \item \textbf{up\_axi} An AXI Lite to uP converter core (see core for documentation).
  \item \textbf{up\_wishbone\_classic} A wishbone classic to uP converter core (see core for documentation).
  \item \textbf{up\_uart} Takes uP bus and coverts it to interface with the RX/TX FIFOs and the AXIS UART (see module documentation for information \ref{Module Documentation}).
\end{itemize}

For register documentation please see up\_uart in \ref{Module Documentation}

\section{Building}

\par
The BUS UART is written in Verilog 2001. It should synthesize in any modern FPGA software. The core comes as a fusesoc packaged core and can be
included in any other core. Be sure to make sure you have meet the dependencies listed in the previous section.

\subsection{fusesoc}
\par
Fusesoc is a system for building FPGA software without relying on the internal project management of the tool. Avoiding vendor lock in to Vivado or Quartus.
These cores, when included in a project, can be easily integrated and targets created based upon the end developer needs. The core by itself is not a part of
a system and should be integrated into a fusesoc based system. Simulations are setup to use fusesoc and are a part of its targets.

\subsection{Source Files}

\input{src/fusesoc/files_axi_lite_uart.tex}

\input{src/fusesoc/files_wishbone_classic_uart.tex}

\input{src/fusesoc/files_up_uart.tex}

\subsection{Targets}

\input{src/fusesoc/targets_axi_lite_uart.tex}

\input{src/fusesoc/targets_wishbone_classic_uart.tex}

\input{src/fusesoc/targets_up_uart.tex}

\subsection{Directory Guide}

\par
Below highlights important folders from the root of BUS UART.

\begin{enumerate}
  \item \textbf{docs} Contains all documentation related to this project.
    \begin{itemize}
      \item \textbf{manual} Contains user manual and github page that are generated from the latex sources.
    \end{itemize}
  \item \textbf{src} Contains source files for the core
  \item \textbf{tb} Contains test bench files for iverilog and cocotb
    \begin{itemize}
      \item \textbf{cocotb} testbench files
    \end{itemize}
\end{enumerate}

\newpage

\section{Simulation}
\par
There are a few different simulations that can be run for this core.

\subsection{iverilog}
\par
iverilog is used for simple test benches for quick verification, visually, of the core.

\subsection{cocotb}
\par
Future simulations will use cocotb. This feature is not yet implemented.

\newpage

\section{Module Documentation} \label{Module Documentation}

\par
up\_uart is the module that integrates the AXIS UART core.
This includes FIFO's that have there inputs/outputs for data tied to registers mapped in the uP bus.
The uP bus is the microprocessor bus based on Analog Devices design. It resembles a APB bus in design,
and is the bridge to other buses BUS UART can use. This makes changing for AXI Lite, to Wishbone to whatever
quick and painless.

\par
axi\_lite\_uart module adds a AXI Lite to uP (microprocessor) bus converter. The converter is
from Analog Devices.

\par
wishbone\_classic\_uart module adds a Wishbone Classic to uP (microprocessor) bus converter. This
converter was designed for Wishbone Classic only, NOT pipelined.

\vspace{15mm}
\par
The next sections document these modules in great detail. up\_uart contains the register map explained, and what the various bits do.

